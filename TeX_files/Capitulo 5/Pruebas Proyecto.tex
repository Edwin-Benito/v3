\section{Pruebas de Proyecto}

Las pruebas de proyecto van más allá de la interfaz y buscan validar el concepto, la estrategia y la viabilidad de la iniciativa Argos en el contexto institucional, asegurando su alineación con las necesidades y capacidades de la universidad.

\subsection{Focus Group con Especialistas}

\subsubsection{Definición}
Un focus group es una técnica de investigación cualitativa que reúne a un pequeño grupo de expertos (entre 3 y 5 personas) para discutir un tema específico bajo la guía de un moderador. A diferencia de una entrevista, fomenta la interacción y el debate entre los participantes.

\subsubsection{Objetivo}
Para el Proyecto Argos, se organizó un focus group con especialistas clave de la UPP (un coordinador académico, el jefe de servicios de TI y un administrador escolar). El objetivo fue \textbf{presentar el proyecto en su totalidad (problema, solución, mockups, plan) y validar su viabilidad técnica, operativa y estratégica} desde una perspectiva institucional, recogiendo su feedback experto.

\subsubsection{Ventajas}
Realizar esta prueba de concepto con especialistas aportó beneficios estratégicos:
\begin{itemize}
	\item \textbf{Identificación Temprana de Riesgos:} Los especialistas ayudaron a identificar posibles desafíos de integración con los sistemas escolares existentes (como Banner) y con las políticas de seguridad de la red universitaria.
	\item \textbf{Obtención de Apoyo (Buy-in):} Presentar el proyecto y tomar en cuenta sus opiniones generó apoyo temprano de figuras clave, lo cual es vital para la futura implementación del proyecto.
	\item \textbf{Sugerencias de Expertos:} Se obtuvieron valiosas recomendaciones sobre normativas de protección de datos que la universidad debe seguir, así como ideas para una posible implementación piloto en un solo departamento.
\end{itemize}

\clearpage
\subsection{Adaptación de Innovaciones}

\subsubsection{Definición}
La adaptación de innovaciones es el proceso de analizar y planificar cómo una nueva tecnología o sistema será adoptado por una comunidad. Se enfoca en comprender y gestionar los factores sociales, culturales y psicológicos que influyen en la aceptación del cambio para asegurar una transición exitosa.

\subsubsection{Objetivo}
El objetivo para Argos es \textbf{diseñar una estrategia proactiva que asegure una transición suave y una alta tasa de adopción} del nuevo sistema por parte de toda la comunidad universitaria. Esto incluye planificar cómo se comunicarán los beneficios, cómo se abordarán las preocupaciones sobre privacidad (especialmente con el reconocimiento facial), y cómo se diseñarán los programas de capacitación para los diferentes roles.

\subsubsection{Ventajas}
Planificar la adaptación de la innovación desde la fase de diseño ofrece ventajas a largo plazo:
\begin{itemize}
	\item \textbf{Reduce la Resistencia al Cambio:} Al anticipar y abordar las preocupaciones de los usuarios (ej. "¿Qué pasará con mis datos biométricos?"), se minimiza la fricción y la oposición durante la implementación.
	\item \textbf{Asegura el Retorno de Inversión (ROI):} Una alta tasa de adopción por parte de los usuarios finales es la única manera de garantizar que los beneficios proyectados del sistema (eficiencia, seguridad, datos precisos) se materialicen.
	\item \textbf{Construye Confianza y Transparencia:} Una comunicación clara y una buena capacitación generan confianza en la nueva tecnología y en la gestión de la institución, demostrando que se toma en cuenta el factor humano.
\end{itemize}
