\section{Evaluación de Condiciones y Necesidades del Proyecto}

\subsection{Técnicas para identificar problemas}

La identificación de problemas es un paso fundamental en el desarrollo de un proyecto, esto permite comprender las necesidades 
reales del contexto, los usuarios y las organizaciones involucradas. A continuación, se describen las técnicas básicas 
utilizadas en esta etapa: Estas técnicas fueron seleccionadas considerando la diversidad de actores involucrados 
(alumnos, personal administrativo, padres de familia, docentes) y la complejidad del sistema propuesto.

\subsubsection*{Análisis de Cuestionarios y Entrevistas}
Durante el mes de junio se aplicaron encuestas y entrevistas presenciales en las instalaciones de la
 Universidad Politécnica de Pachuca, principalmente en horarios de entrada y salida escolar. El objetivo de esta técnica 
 fue identificar necesidades, dificultades y expectativas de los usuarios clave respecto al sistema propuesto. 
 Su utilidad radica en que permitió obtener información directa y variada de alumnos, docentes y personal administrativo, 
 facilitando la detección de problemas y oportunidades de mejora.

\subsubsection*{Observación Directa}
Esta técnica se llevó a cabo durante junio en los accesos principales de la institución, en los horarios de mayor afluencia. 
El objetivo fue analizar el comportamiento real de los usuarios y los procesos actuales de registro de entrada.
 La observación directa resultó útil para detectar problemas no expresados verbalmente y observar el flujo real de personas y 
 los tiempos de espera.

\subsubsection*{Consulta con Expertos}
Durante el mes de junio se realizaron reuniones presenciales y virtuales con especialistas en tecnologías de la información 
y gestión escolar. El objetivo fue validar la viabilidad técnica y operativa de la solución propuesta. 
Esta técnica fue útil porque aportó recomendaciones sobre integración tecnológica, seguridad y
 mejores prácticas, anticipando posibles retos y permitiendo ajustar el diseño conceptual del sistema.





% Para que las subsubsecciones aparezcan numeradas y en el índice
\setcounter{secnumdepth}{3}
\setcounter{tocdepth}{3}
