\section{Identificación y Análisis del Problema}
El problema central que enfrenta la Universidad Politécnica de Pachuca es la \textbf{saturación en los puntos de acceso y el registro de asistencia}. Actualmente, estos procesos se realizan de forma manual, lo que provoca demoras considerables para la comunidad universitaria y afecta negativamente el inicio puntual de las clases. Esta situación no solo genera inconvenientes y desorden, sino que también dificulta un registro preciso y en tiempo real, impactando la seguridad y la experiencia general dentro del campus.

\subsection{Descripción del Problema}
La Universidad Politécnica de Pachuca experimenta una \textbf{congestión significativa en sus accesos} debido a la falta de un sistema automatizado para el registro de entrada y asistencia. Los procesos manuales actuales, como el uso de listas impresas o la verificación visual de credenciales, no son escalables para la cantidad de estudiantes y personal. Esta práctica también abre la posibilidad de que personas externas a la universidad ingresen sin autorización, incrementando los riesgos para la seguridad del campus. Esto resulta en \textbf{filas largas y retrasos}, lo que retrasa el comienzo de las actividades académicas. Además, el método de registro de asistencia en el aula es ineficiente, lo que impide a los docentes y a la administración tener un control preciso y oportuno sobre la presencia de los alumnos. En resumen, el sistema actual es lento, propenso a errores y genera una experiencia negativa para toda la comunidad.

\subsection{Causas del Problema}
Las tres causas principales que originan la problemática son:
\begin{itemize}
    \item \textbf{Procesos de registro manuales:} La dependencia de métodos tradicionales para el registro de entrada y asistencia, como el llenado de listas de papel o la verificación manual de credenciales, es la causa fundamental de la saturación. Estos procesos son ineficaces y consumen tiempo valioso, incapaces de manejar el flujo de personas de manera ágil.
    \item \textbf{Falta de tecnología de identificación:} La ausencia de una solución tecnológica moderna, como el reconocimiento facial o códigos QR, impide un registro rápido y eficiente. Al no haber herramientas automáticas, se requiere una intervención humana constante que ralentiza todo el proceso.
    \item \textbf{Sistema de gestión de asistencia ineficiente:} No existe una plataforma digital que permita a los docentes registrar la asistencia de forma sencilla y en tiempo real. Esta carencia no solo dificulta el seguimiento académico, sino que también contribuye al desorden general al inicio de las clases, prolongando el tiempo que se tarda en comenzar las actividades programadas.
\end{itemize}