\section{Análisis de Costos y Esfuerzo (Método COCOMO)}

Para estimar el esfuerzo y el tiempo del proyecto \textit{Argos}, hemos utilizado el modelo COCOMO (Modelo Constructivo de Costos) en su versión intermedia. Este método se basa en el tamaño del software y en una serie de factores de costo para generar una estimación realista.

\subsection{Estimación del Tamaño del Software (KLOC)}
Debido a que el proyecto se encuentra en una etapa de planeación, el tamaño del software se ha estimado en *25,000 líneas de código (25 KLOC)*. Esta cifra se considera apropiada para una solución integral que incluye un sistema de reconocimiento facial, una aplicación móvil, una página web y la infraestructura de base de datos para el almacenamiento seguro de la información.

\subsection{Cálculo del Esfuerzo y Tiempo del Proyecto}
El modelo COCOMO intermedio se basa en dos ecuaciones principales:
\begin{itemize}
    \item *Esfuerzo (E):* Se mide en persona-mes.
    $E = a \times (\mathrm{KLOC})^{b} \times \mathrm{EAF}$
    \item *Tiempo de Desarrollo (TDEV):* Se mide en meses.
    $TDEV = c \times (\mathrm{E})^{d}$
\end{itemize}
Para un proyecto de tipo \textbf{semi-independiente} (considerando la naturaleza del proyecto que combina elementos de innovación con requisitos bien definidos), los coeficientes son: $a=3.0$, $b=1.12$, $c=2.5$ y $d=0.35$.

\subsubsection{Factores de Costo (EAF - \textit{Effort Adjustment Factor})}
Hemos asignado valores a los factores de costo (EAF) con base en la información del proyecto y sus características.

\begin{enumerate}
    \item \textbf{Atributos del Producto}
    \begin{itemize}
        \item Fiabilidad requerida del software (\textit{RELY}): 1.15 (Alta)
        \item Tamaño de la base de datos (\textit{DATA}): 1.00 (Nominal)
        \item Complejidad del producto (\textit{CPLX}): 1.29 (Muy Alta)
    \end{itemize}
    \item \textbf{Atributos del Hardware}
    \begin{itemize}
        \item Restricciones de tiempo de ejecución (\textit{TIME}): 1.00 (Nominal)
        \item Restricciones de memoria principal (\textit{STOR}): 1.00 (Nominal)
        \item Volatilidad del entorno de la máquina virtual (\textit{VIRT}): 0.94 (Baja)
        \item Tiempo de respuesta del sistema (\textit{TURN}): 1.07 (Alta)
    \end{itemize}
    \item \textbf{Atributos del Personal}
    \begin{itemize}
        \item Capacidad del analista (\textit{ACAP}): 1.00 (Nominal)
        \item Experiencia en la aplicación (\textit{AEXP}): 1.00 (Nominal)
        \item Capacidad del programador (\textit{PCAP}): 1.00 (Nominal)
        \item Experiencia en la máquina virtual (\textit{VEXP}): 1.00 (Nominal)
        \item Experiencia en el lenguaje de programación (\textit{LEXP}): 1.00 (Nominal)
    \end{itemize}
    \item \textbf{Atributos del Proyecto}
    \begin{itemize}
        \item Uso de herramientas de software (\textit{TOOL}): 1.00 (Nominal)
        \item Cumplimiento de plazos del proyecto (\textit{SCED}): 1.11 (Alta)
    \end{itemize}
\end{enumerate}

El \textbf{EAF} se calcula multiplicando todos estos factores de costo:
$EAF = 1.15 \times 1.00 \times 1.29 \times 1.00 \times 1.00 \times 0.94 \times 1.07 \times 1.00 \times 1.00 \times 1.00 \times 1.00 \times 1.00 \times 1.00 \times 1.11 \approx 1.83$
 3. Resultados del Cálculo
Aplicando las fórmulas con los valores estimados, obtenemos:
\begin{itemize}
    \item \textbf{Cálculo del Esfuerzo (E):}
    $E = 3.0 \times \left(25\right)^{1.12} \times 1.83 \approx 199.1$ persona-mes.
    \item \textbf{Cálculo del Tiempo de Desarrollo (TDEV):}
    $TDEV = 2.5 \times \left(199.1\right)^{0.35} \approx 18.2$ meses.
\end{itemize}

4. Conclusiones
Los resultados del modelo COCOMO intermedio indican que el proyecto \textit{Argos} requerirá:
\begin{itemize}
    \item Un \textbf{esfuerzo total} de aproximadamente \textbf{199.1 persona-mes}.
    \item Un \textbf{tiempo de desarrollo} de \textbf{18.2 meses}.
    \item Un \textbf{equipo promedio} de $\frac{199.1\ \mathrm{persona-mes}}{18.2\ \mathrm{meses}} \approx 11$ personas.
\end{itemize}
Esta estimación valida la complejidad y el alcance del proyecto, demostrando que para su correcta ejecución se necesita un equipo de desarrollo más grande y un cronograma más extenso que el planteado en el Diagrama de Gantt inicial. La duración de 105 días hábiles (aproximadamente 5 meses) mencionada en la planeación se enfoca en las primeras etapas de análisis y diseño, no en la implementación completa del sistema.