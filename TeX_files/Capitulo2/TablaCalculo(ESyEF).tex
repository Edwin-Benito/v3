\subsection{Cálculos de ES y EF (Paso Hacia Adelante)}

El "Paso Hacia Adelante" (Forward Pass) se utiliza para determinar la Fecha de Inicio Temprana (ES - Early Start) y la Fecha de Finalización Temprana (EF - Early Finish) para cada actividad, ahora en días hábiles (1 semana = 5 días).

\begin{itemize}
    \item \textbf{ES (Fecha de Inicio Temprana):} La fecha más temprana en que una actividad puede comenzar, asumiendo que todas sus predecesoras se han completado.
    \item \textbf{EF (Fecha de Finalización Temprana):} La fecha más temprana en que una actividad puede terminar. Se calcula como $EF = ES + Duración$.
\end{itemize}

\begin{table}[htbp]
  \centering
  \caption{Cálculos de ES y EF (Paso Hacia Adelante)}
  \renewcommand{\arraystretch}{1.3}
  \setlength{\tabcolsep}{7pt}
  \rowcolors{2}{gray!10}{white}
  \begin{tabularx}{\linewidth}{>{\centering\arraybackslash}p{1.2cm} X >{\centering\arraybackslash}p{2.2cm} >{\centering\arraybackslash}p{2.2cm} >{\centering\arraybackslash}p{1.8cm} >{\centering\arraybackslash}p{1.8cm}}
    \toprule
    \rowcolor{gray!30} \textbf{ID} & \textbf{Actividad} & \textbf{Duración (Días)} & \textbf{Predecesoras} & \textbf{ES} & \textbf{EF} \\
    \midrule
      A & Análisis del Problema & 20 & --- & 0 & 20 \\
        B & Recolección de Requerimientos & 20 & A & 20 & 40 \\
        C & Diseño de Interfaz de Usuario & 30 & B & 40 & 70 \\
        D & Revisión de Documentación & 5 & C & 70 & 75 \\
        E & Pruebas Usabilidad & 10 & D & 75 & 85 \\
        F & Validación de Entregables & 10 & D & 75 & 85 \\
        G & Presentación del Proyecto & 5 & E, F & 85 & 90 \\
    \bottomrule
  \end{tabularx}
  \vspace{0.7em}
  \noindent\textbf{Duración real (ES/EF):} 10 semanas hábiles (50 días) en 11 semanas calendario.\\
  \noindent\textbf{Esfuerzo total:} 100 días hábiles.\\
  \noindent\textbf{Nota:} 100 días = esfuerzo; 10 semanas = duración por trabajo en paralelo.
\end{table}