\subsection{Cálculos de LS y LF (Paso Hacia Atrás)}

El "Paso Hacia Atrás" (Backward Pass) se utiliza para determinar la Fecha de Finalización Tardía (LF - Late Finish) y la Fecha de Inicio Tardía (LS - Late Start) para cada actividad sin retrasar la fecha de finalización del proyecto, ahora en días hábiles (1 semana = 5 días).

\begin{itemize}
    \item \textbf{LF (Fecha de Finalización Tardía):} La fecha más tardía en que una actividad puede terminar sin retrasar el proyecto.
    \item \textbf{LS (Fecha de Inicio Tardía):} La fecha más tardía en que una actividad puede comenzar sin retrasar el proyecto. Se calcula como $LS = LF - Duración$.
\end{itemize}

\begin{table}[htbp]
  \centering
  \caption{Cálculos de LS y LF (Paso Hacia Atrás)}
  \renewcommand{\arraystretch}{1.3}
  \setlength{\tabcolsep}{7pt}
  \rowcolors{2}{gray!10}{white}
  \begin{tabularx}{\linewidth}{>{\centering\arraybackslash}p{1.2cm} X >{\centering\arraybackslash}p{2.2cm} >{\centering\arraybackslash}p{2.2cm} >{\centering\arraybackslash}p{1.8cm} >{\centering\arraybackslash}p{1.8cm}}
    \toprule
    \rowcolor{gray!30} \textbf{ID} & \textbf{Actividad} & \textbf{Duración (Días)} & \textbf{Sucesores} & \textbf{LF} & \textbf{LS} \\
    \midrule
      G & Presentación del Proyecto & 5 & --- & 90 & 85 \\
        E & Pruebas Usabilidad & 10 & G & 85 & 75 \\
        F & Validación de Entregables & 10 & G & 85 & 75 \\
        D & Revisión de Documentación & 5 & E, F & 75 & 70 \\
        C & Diseño de Interfaz de Usuario & 30 & D & 70 & 40 \\
        B & Recolección de Requerimientos & 20 & C & 40 & 20 \\
        A & Análisis del Problema & 20 & B & 20 & 0 \\
    \bottomrule
  \end{tabularx}
\end{table}