\subsection{Cálculo de Holgura y Ruta Crítica}

La holgura (también conocida como flotación o margen) es la cantidad de tiempo que una actividad puede retrasarse sin afectar la fecha de finalización del proyecto. Se calcula como $Holgura = LS - ES$ o $Holgura = LF - EF$, ahora en días hábiles (1 semana = 5 días).

Las actividades con holgura cero son las que forman la Ruta Crítica. Un retraso en cualquiera de estas actividades impactará directamente en la fecha de finalización de todo el proyecto.

\begin{table}[htbp]
  \centering
  \caption{Cálculo de Holgura y Ruta Crítica}
  \renewcommand{\arraystretch}{1.3}
  \setlength{\tabcolsep}{6pt}
  \rowcolors{2}{gray!10}{white}
  \begin{tabularx}{\linewidth}{>{\centering\arraybackslash}p{1.2cm} X >{\centering\arraybackslash}p{1.2cm} >{\centering\arraybackslash}p{1.2cm} >{\centering\arraybackslash}p{1.2cm} >{\centering\arraybackslash}p{1.2cm} >{\centering\arraybackslash}p{1.5cm} >{\centering\arraybackslash}p{2.2cm}}
    \toprule
    \rowcolor{gray!30} \textbf{ID} & \textbf{Actividad} & \textbf{ES} & \textbf{EF} & \textbf{LS} & \textbf{LF} & \textbf{Holgura} & \textbf{¿En Ruta Crítica?} \\
    \midrule
    A & Análisis del Problema & 0 & 20 & 0 & 20 & 0 & Sí \\
    B & Recolección de Requerimientos & 20 & 40 & 20 & 40 & 0 & Sí \\
    C & Diseño de Interfaz de Usuario & 40 & 70 & 40 & 70 & 0 & Sí \\
    D & Revisión de Documentación & 70 & 75 & 70 & 75 & 0 & Sí \\
    E & Pruebas Usabilidad & 75 & 85 & 75 & 85 & 0 & Sí \\
    F & Validación de Entregables & 75 & 85 & 75 & 85 & 0 & Sí \\
    G & Presentación del Proyecto & 85 & 90 & 85 & 90 & 0 & Sí \\
    \bottomrule
  \end{tabularx}
  \vspace{0.7em}
  \noindent\textbf{Ruta Crítica del Proyecto:} Todas las actividades tienen holgura cero, por lo que todas son críticas. Cualquier retraso impacta la fecha de finalización.

  \noindent\textbf{Ruta Crítica:} A $\rightarrow$ B $\rightarrow$ C $\rightarrow$ D $\rightarrow$ E $\rightarrow$ G y A $\rightarrow$ B $\rightarrow$ C $\rightarrow$ D $\rightarrow$ F $\rightarrow$ G.

  \noindent\textbf{Duración real:} 10 semanas hábiles (50 días) en 11 semanas calendario.\\
  \noindent\textbf{Esfuerzo total:} 100 días hábiles.\\
  \vspace{0.5em}
  \noindent\textbf{Nota:} 100 días = esfuerzo; 10 semanas = duración por trabajo en paralelo.
\end{table}