\section{Normativas}

Para asegurar la calidad y seguridad en el Proyecto Argos, se adoptaron las siguientes normativas internacionales clave. Estas guían el desarrollo, la gestión y la protección de la información en el sistema:

\subsubsection*{ISO 9001:2015 – Sistemas de Gestión de la Calidad}
Establece un sistema de gestión de calidad basado en la mejora continua y la satisfacción del cliente. En Argos, se aplica para documentar procesos, auditar actividades clave y asegurar que cada entregable cumpla los requisitos funcionales y no funcionales definidos desde el inicio del proyecto.

\subsubsection*{ISO/IEC 25000 – Calidad del Producto de Software (SQuaRE)}
Proporciona criterios para evaluar la calidad del software en funcionalidad, usabilidad y eficiencia. Permite definir atributos medibles, estandarizar pruebas con usuarios finales (alumnos, docentes y padres de familia) y detectar posibles mejoras antes de la implementación final.

\subsubsection*{ISO/IEC 27001 – Seguridad de la Información}
Se enfoca en la gestión de la seguridad de la información, esencial para proteger los datos biométricos y registros de asistencia. Incluye prácticas de gestión de riesgos, uso de cifrado y autenticación segura, así como políticas de respaldo y recuperación ante incidentes.

\subsubsection*{ISO/IEC 29110 – Ingeniería de Software para PYMES y Equipos Pequeños}
Pensada para proyectos pequeños, facilita la documentación simple, la definición clara de roles y responsabilidades, y la adopción de procesos ágiles y estructurados en las fases de diseño, codificación, pruebas y validación del sistema.

\subsubsection*{Conclusión}
La integración de estas normativas fortalece la calidad y seguridad del sistema, asegurando un producto confiable, seguro y alineado con estándares internacionales. El Encargado de Calidad supervisa su aplicación y verifica, mediante auditorías internas, que cada práctica esté alineada con estos lineamientos.
